\documentclass[usenames,dvipsnames,handout]{beamer}
\usepackage{amsmath} % Improves structure of typed out maths
\usepackage{mathtools} % Improves upon deficiencies of amsmath package
\usepackage{amssymb} % Adds some handy symbols to use.
\usepackage{amsthm} % Adds some neat formulas to use, e.g. \begin{proof} etc.

\usepackage{media9} % Allows for the inclusion of GIF files.

% Below are other packages I suggest you add
%\usepackage{geometry} % Default page margins can be altered.
\usepackage{microtype} % Improves spacing between letters.
%\usepackage{booktabs} % Improves tables. Can now create without vertical separators.
%\usepackage{array} % Includes more options for arrays

%\usepackage{graphicx} % Add images to your document
%\usepackage{xcolor}
%\usepackage{cleveref} % Better cross-referencing
%\usepackage{hyperref} % For adding hyperlinks
%\usepackage{fancyhdr} % Customise headers & footers in document

%\usepackage[parfill]{parskip}

%\usepackage{enumerate}
\usepackage{tikz, pgf, pgfplots}
\pgfplotsset{compat=1.15}

\usetheme{Warsaw}
%\useinnertheme{rounded}
\useoutertheme{WarsawNoListHeader}


\definecolor{quantsocpurple1}{RGB}{150, 61, 249}
\definecolor{quantsocpurple2}{RGB}{90, 1, 191}
\definecolor{quantsocpurple3}{RGB}{238, 227, 255}
\definecolor{quantsocalerted}{RGB}{160, 31, 31}
\definecolor{quantsocexample}{RGB}{17, 79, 20}
%\definecolor{alerttextcolor}{RGB}{147, 8, 216}

\setbeamercolor{palette primary}{bg=quantsocpurple1,fg=white}
\setbeamercolor{palette secondary}{bg=quantsocpurple1,fg=white}
\setbeamercolor{palette tertiary}{bg=quantsocpurple1,fg=white}
\setbeamercolor{palette quaternary}{bg=quantsocpurple1,fg=white}
\setbeamercolor{structure}{fg=quantsocpurple1} % itemize, enumerate, etc
\setbeamercolor{section in toc}{fg=quantsocpurple1} % TOC sections

% Override palette coloring with secondary
\setbeamercolor{subsection in head/foot}{bg=quantsocpurple2,fg=white}

\setbeamercolor{block title}{bg=quantsocpurple1}
\setbeamercolor{block title alerted}{bg=quantsocalerted}
\setbeamercolor{block title example}{bg=quantsocexample}

%\setbeamercolor{alerted text}{fg=alerttextcolor}

\AtBeginEnvironment{alertblock}{\setbeamercolor{itemize item}{fg=quantsocalerted}}
\AtBeginEnvironment{alertblock}{\setbeamercolor{itemize subitem}{fg=quantsocalerted}}
\AtBeginEnvironment{alertblock}{\setbeamercolor{itemize subsubitem}{fg=quantsocalerted}}
\AtBeginEnvironment{alertblock}{\setbeamercolor{enumerate item}{fg=quantsocalerted}}
\AtBeginEnvironment{alertblock}{\setbeamercolor{enumerate subitem}{fg=quantsocalerted}}
\AtBeginEnvironment{alertblock}{\setbeamercolor{enumerate subsubitem}{fg=quantsocalerted}}

\AtBeginEnvironment{exampleblock}{\setbeamercolor{itemize item}{fg=quantsocexample}}
\AtBeginEnvironment{exampleblock}{\setbeamercolor{itemize subitem}{fg=quantsocexample}}
\AtBeginEnvironment{exampleblock}{\setbeamercolor{itemize subsubitem}{bg=quantsocexample!80, fg=quantsocexample}}
\AtBeginEnvironment{exampleblock}{\setbeamercolor{enumerate item}{fg=quantsocexample}}
\AtBeginEnvironment{exampleblock}{\setbeamercolor{enumerate subitem}{fg=quantsocexample}}
\AtBeginEnvironment{exampleblock}{\setbeamercolor{enumerate subsubitem}{fg=quantsocexample}}

%\setbeamertemplate{footline}[frame number]
\title{Point Process Modeling of Limit Order Books}
\subtitle{}

\author[] % (optional)
{Oden Petersen}

\date{}
\logo{\includegraphics[scale=0.025]{qslogo.jpeg}}
\institute{UNSW Mathematics and Statistics}

%\AtBeginSection[]
%{
%  \begin{frame}
%    \frametitle{Today's plan}
%    \tableofcontents[currentsection]
%  \end{frame}
%}
%------------------------------------------------------------

%\setbeamercovered{transparent}

\begin{document}

\frame{\titlepage}

\begin{frame}{Stock Prices}
	Graph
	normally modeled with time series models. eg autoregression, state models, garch
	these models have had some success but don't describe the process of price formation from first principles
\end{frame}

\begin{frame}{Limit Order Book}
	Markets are venues for buyers and sellers to come together to exchange financial assets
	Limit order book diagram
	Order components
	Modeling individual order arrivals, cancellations, and trades allows us to simulate the evolution of the limit order book
\end{frame}

\begin{frame}{Order Book Events}
	For order insertions, cancellations we need to model size, price, side, and time.
	First three can be modeled with standard time series tools. But modeling arrival time process is nontrivial. Some parts of the day have lots of arrivals, others don't.
	graph of intraday activity bucketed by minute
\end{frame}

\begin{frame}{Inhomogenous Poisson Process Model}
	graph of intraday activity bucketed by minute
	overlay maxlikelihood spline intensity

	counting process overlaid with integral $\bar{\Lambda}$ of intensity
\end{frame}

\begin{frame}{Likelihood function for Point Process Models}
	Explain why the loglikelihood formula is reasonable
\end{frame}

\begin{frame}{Residuals}
	Why integral of intensity should be iid exponential
	Show that it isnt $\to$ motivate autoregressive intensity
\end{frame}

\begin{frame}{Hawkes Processes}
	Formula for intensity
	kernel is linear combination of exponential functions. can approximate analytic functions on $[0,t]$ uniformly well with enough exponential components by weierstrass approximation theorem.
	Can treat events as coming from a mixture of multiple point processes. As is common with mixture models, use an EM algorithm to fit.
	Explain how to get branching matrix.
\end{frame}

\begin{frame}{State Dependence}
	Book imbalance vs time until next buy event / time until next sell event
	Motivate state dependence
	Formula for state dependence - discrete and continuous
	Maybe some more details on book imbalance
\end{frame}

\begin{frame}{Point Process Simulation}
	Detail Ogata algorithm
	Original paper uses markov state switching. I can use time series model for orders and limit order book simulation to make this a bit more sophisticated.
\end{frame}

\begin{frame}{Model Results}
	Parametric bootstrap
	IC by model
	crossvalidation loglikelihood
	actual vs expected counts
	residual distribution and autocorrelation and ks pval table
	replication number and distance to average parent for various kernels
\end{frame}

\begin{frame}{Price Impact Function}
	Graph of impact vs size
	VWAP vs TWAP vs 
\end{frame}

\section{appendix}

\begin{frame}{EM and branching matrix details}
\end{frame}

\begin{frame}{Future directions}
	Options markets
	Metaorder modelling via random kernels each day
\end{frame}

\end{document}
